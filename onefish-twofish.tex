\documentclass[11pt,a5paper]{book}
\usepackage[utf8]{inputenc}
\usepackage{amsmath}
\usepackage{amsfonts}
\usepackage{amssymb}
\usepackage{graphicx}
\usepackage[super]{nth}
\usepackage{float}

\title{One Fish, Two Fish}
\author{Marcel Gietzmann-Sanders}
\date{}
\setcounter{tocdepth}{1}
\begin{document}
\maketitle
\tableofcontents
\newpage
\chapter{How to Digest a Brick}

\section{Well What's the Point?}

Quantitative fisheries science (hereafter referred to just as fisheries modeling) is all about answer the following question:
\newline

\hangafter=0 \hangindent=1cm \noindent How can a fishery bring the maximum long term benefits to society?
\newline

Even in just this question there is a lot to unpack. What's a fishery? What do we mean by maximum? What's does long term mean and how do we quantify benefit? And who is this society? Also what kinds of answers count for the "How" that starts that whole sentence? 
\newline

While no answer to these kinds of questions is going to ever be complete or 100\% right let's at least get a sense of what flavor the answer might be. Let's take each component (shown in \textbf{bold}) in turn.
\newpage

\noindent \rule{\textwidth}{0.5pt} 
\noindent How can a \textbf{fishery} bring the maximum long term benefits to society?
\newline
\rule{\textwidth}{0.5pt} 
\vspace{5pt}

A fishery can be a lot of things. Fishers use all sorts of different kinds of tackle - nets, hooks, trawls, etc. - to catch all manner of sea creatures - fish, turtles, crustaceans, etc. Those creatures live all over the globe, have all sorts of species, populations, and communities, spawn in some areas, forage in others, and so on. Likewise the fishers who catch them operate in all kinds of ways at all kinds of times in all sorts of places. So clearly a fishery can mean a lot of things. It could mean a specific type of tackle used at a certain time of year in a specific place to catch a specific thing, it could mean anyone who tries to take a specific species in a specific area, or it could mean anyone fishing in a specific ecosystem. Yikes! That's not much of a definition. Herein we're going to designate fisheries by the species populations that they affect. This gives us the freedom to either choose to concern ourselves with a single species at a time or deal with ecosystems as a whole - the only difference is how many species we consider. 
\newpage

\noindent \rule{\textwidth}{0.5pt} 
\noindent How can a fishery bring the maximum long term \textbf{benefits} to society?
\newline
\rule{\textwidth}{0.5pt} 
\vspace{5pt}

This one is also ridiculously tricky. Benefits from seafood can be nutritional, cultural, financial, and biological (amongst other things). They can make or break communities, they can be necessary for raising other farmed fish or fertilizing farms. Some fisheries have important medical applications. And all fisheries are subject to the creativity of entrepreneurs - it wasn't so long ago that lobster was prison food. So when quantifying benefit the answer is \textit{absolutely} specific to the fishery in question. As much as folks would like to provide some singular answer, the fact is each fishery has a whole load of different values. \textit{Know your fishery}.
\newpage

\noindent \rule{\textwidth}{0.5pt} 
\noindent How can a fishery bring the maximum \textbf{long term} benefits to society?
\newline
\rule{\textwidth}{0.5pt} 
\vspace{5pt}

Whatever the benefits accrued the hope is that those benefits remain with us over the long term - i.e. that they are sustainable benefits. Fishing a region to death only means that short term benefits rob us from long term sustained (and therefore much larger) benefits. So by long term we mean sustainable. 
\newpage

\noindent \rule{\textwidth}{0.5pt} 
\noindent How can a fishery bring the \textbf{maximum} long term benefits to society?
\newline
\rule{\textwidth}{0.5pt} 
\vspace{5pt}

We already mentioned that there are loads of benefits of all different kinds that can come from a specific fishery. Thing is that depending on how we manage our fishery the relatively scales of those benefits can change dramatically. For example a lot of fish vary in economic value by weight. If you fillet is too small you're not interested in buying it, if it's way to large you may also not be all that interested. Which means that there's a "sweet spot" for when to catch the fish. Catch mostly small fish and you drop the value of the fishery (in this super simplified sense), catch too many large fish and you do the same. So if you manage your fishery to catch more of the fish in the middle you'll increase the value of the fishery quite a bit. 
\newline

This is what is meant by "maximum" - finding creative ways to increase the benefit rendered to society by the fishery. 
\newline

Now note that because there are a load of different kinds of benefits and usually you don't get to increase all benefits without affecting others there's going to be a need to set priorities or weights for each of these benefits so you can know how to balance improvements in one against losses in another - this is going to lead you to a single objective but remember that this objective will be superbly specific to \textit{your} fishery.
\newpage

\noindent \rule{\textwidth}{0.5pt} 
\noindent \textbf{How} can a fishery bring the maximum long term benefits to society?
\newline
\rule{\textwidth}{0.5pt} 
\vspace{5pt}

Alright there are a lot of ways to be creative about all of this. Which means there are a lot of possible hows. Some of these seem pretty relevant to fisheries management - changing the selectively of gear, changing fishing limits, setting up marine protected areas - however others don't seem like they belong to the management "jurisdiction" - opening a restaurant to introduce new recipes and raise the economic value of a fish, changing how the supply chain for fish works to reduce costs in production. So in order to not boil the ocean, what counts as a how here?
\newline

This is obviously a point of opinion, so I'm not going to pretend like my answer is the "right" one (one shouldn't go about bounding creativity wholesale) however I'll give my best shot because I think it's at least somewhat helpful. To my mind the "how" that fisheries modeling focuses on is in defining actions (or inactions) the fishers themselves can take when out fishing. So gear selection, quotas, adjustments to bycatch are all fair game, whereas general fish product production and consumption entrepreneurship are out of scope. 
\newpage

\noindent \rule{\textwidth}{0.5pt} 
\noindent  How can a fishery bring the maximum long term benefits to \textbf{society}?
\newline
\rule{\textwidth}{0.5pt} 
\vspace{5pt}

This is a question with an obvious answer that probably should be paid more attention to. Society is \textit{all} of us. It's not just mega-corps, it's not just the town near the fish that's been fishing there for centuries, and its not the consumer either. It is \textit{everyone} and making sure \textit{all} are represented is of utmost importance. 
\newpage

With all of this mind we can reconstruct our question as the following:
\newline

\hangafter=0 \hangindent=1cm \noindent \textbf{Fisheries modeling is about finding the set of fishing practices required to maximize the the sustainable benefit to all.}
\newline

Alright so how do we go about doing this creative optimization? That's what the next section is about. 
\newpage

\section{The 30,000 Foot View}

\noindent \rule{\textwidth}{0.5pt} 
\begin{figure}[h!] 
  \includegraphics[width=\linewidth]{drawings/high_level_models.png}
  \caption{Models}
  \label{fig:high_level_models}
\end{figure}
\newline
\rule{\textwidth}{0.5pt} 
\vspace{5pt}

The first thing we need is a series of models that will allow us to... well... model how changing fishing practice will effect our overall benefit. These models divide into, broadly speak, three different categories. 
\newline

The benefits model is how we derive, from a specific situation, what the expected long term benefit would be. For example if the \textit{only} benefit is economic first sale benefit based on weight then the relationship between weight and cost would be one aspect of our model. This could then be applied to the expected catch each year (structured by weight of course) to then produce an expected value of the fishery year over year which could then be integrated to give us the long term benefit. In general however benefits models are far more complicated than this as, as we've already mentioned, there's a lot more to benefit that just value per weight class.
\newline

The second model we need is the population model. This describes how the populations in question change in response to the environment around them (which can include the fishers). A simple example would be a model that predicts the amount of recruitment (new fish babies) to the population as a function of the current population along with a model of how quickly the fish die off from natural mortality as the years go by. These two then allow us to predict how the population will change over time. However these models can get far more complex than just this (and most often do). 
\newline

In the middle of these two models we need a catch model. This is the model that describes how our choices in managing the fishery effect the underlying populations. Often times this model is as simple as a relationship between fishing effort and fishing mortality with some notion of gear selectivity included (gear selectivity being the fact that a specific kind of gear will typically catch a particular kind of age or weight of fish). Without this model there's no way to connect our benefit model to the population model. It's also the model that takes the parameters of fishery action that we're ultimately using as the knobs to tune to optimize our overall sustained benefit to society. 
\newpage

\noindent \rule{\textwidth}{0.5pt} 
\begin{figure}[h!] 
  \includegraphics[width=\linewidth]{drawings/high_level_instrumentation.png}
  \caption{Instrumentation}
  \label{fig:high_level_instrumentation}
\end{figure}
\newline
\rule{\textwidth}{0.5pt} 
\vspace{5pt}

We've been talking about models quite flippantly so far. But all models are approximations of reality that need to be fit. What does fitting mean? Let's take a look at a simple example. One thing that fisheries scientists often have to model is the relationship between age (usually in years $t$) to body length $L$. One such model is the Von Bertalanffy growth curve:

$$L = L_{\infty}(1-e^{-Kt})$$


As with all models, this model divides into three parts: input variables - $t$, output variables - $L$, and parameters - $L_{\infty}$ and $K$. The input variables are things we'd measure and the output variables are what we want to predict but in order to do so we need to choose specific values for the parameters. Well you'd start with measurements like in Fig. \ref{fig:length_measurements_by_year}.


\begin{figure}[H] 
  \includegraphics[scale=0.35]{notebooks/Fitting/new_measurements.png}
  \caption{Average Length Measurements by Year}
  \label{fig:length_measurements_by_year}
\end{figure}

Now the question is how do we find the parameters that best "fit" this data? Well from our equation we know that eventually (as $t \rightarrow \infty$) our length will go to $L_\infty$. So looking at this data we can guess that $L_\infty = 0.7$.  
\newline

However, what value should we assign to $K$? Well what we can do is try several values and see which fits best.

\begin{figure}[H] 
  \includegraphics[width=\linewidth]{notebooks/Fitting/fit_lines.png}
  \caption{Attempting to Fit $K$}
  \label{fig:fitting_K}
\end{figure}

Fig. \ref{fig:fitting_K} shows us the predicted length vs year relationship for three values of $K$, 0.2, 0.3, and 0.4. By just visual inspection we can see that 0.2 doesn't grow fast enough whereas 0.4 grows to quickly, so 0.3 is probably a reasonable fit.
\newline

In general there are automated mechanisms for doing this but the lesson remains the same - to fit a model you need measurements that can act as ground truth for your fitting (or training) procedure. And that means that for each and every one of our models we're going to need instrumentation to generate that ground truth data. 
\newpage

\noindent \rule{\textwidth}{0.5pt} 
\begin{figure}[h!] 
  \includegraphics[width=\linewidth]{drawings/high_level_optimization.png}
  \caption{Optimization}
  \label{fig:high_level_optimization}
\end{figure}
\newline
\rule{\textwidth}{0.5pt} 
\vspace{5pt}

Alright so we've got ourselves some models that have been fit using the data from the corresponding instrumentation. Now what? Well now we're in a position to use these models to determine what the best fishing practice looks like. Let's take a real simple example to illustrate this.
\newline

All optimizations begin with clearly parametrizing the boundaries of the management practice. In our case, to keep things simple, we're going to assume that we're just directly controlling fishing mortality $F$. Now the basic population model we'll use will assume that if $Z$ is the overall instantaneous mortality rate for fish, in other words we have:

$$\frac{dN}{dt}=-ZN$$

which more or less reads as - we loose population at a rate of $Z\%$ of the overall population size. This is a pretty standard exponential growth equation whose solution is:

$$N = N_0 e^{-Zt}$$

Next we'll also assume we're in an equilibrium state such that the development of one age group describes the whole population. Therefore in year $t_y$ the population size:

$$N_y = N_0 e^{-Zt_y}$$

Next we're going to make the assumption that our $Z=M+F$ where $M$ is the natural mortality (fit with our instrumentation data). 
\newline

This gives our population model. How about the catch model? Well we know that $F/Z$ represents the portion of the mortality that ends up as catch. So if the overall fish mortality in an age class $D_y$ is given by:

$$D_y = N_{y-1} - N_y =N_0(e^{-Zt_{y-1}}- e^{-Zt_y})$$

Then the catch for that year class would be $C_y$:

$$C_y = \frac{F}{Z} D_y = \frac{FN_0}{Z}(e^{-Zt_{y-1}}- e^{-Zt_y})$$

Alright we've got our catch model, so what about our benefits model? Once again we're going to make a couple of really sweeping assumptions. First we're going to assume that no matter what we do the 0th year class (i.e. larval class) is always the same size. This is obviously a wild assumption because if $F=1$ there'd by absolutely no fish to create those babies. However we're going to assume that $F=1$ is impossible and therefore not a worry for us. The other assumption we're going to make is that the value of a fish is equal to its length cubed. Basically we're assume that weight and length are connected by the same relationship and that overall catch weight is what matters. Obviously then we need a length model. We'll use a Von Bertalanffy growth curve:

$$L_y = L_{\infty}(1-e^{-Kt_y})$$

Therefore the value of a catch from a specific year class is:

$$V_y = L_y^3C_y$$

And the total value over the whole class is:

$$V = \sum_y V_y$$

If we expand this completely we get:

$$V = \sum_y \frac{FN_0}{F+M}(e^{-(F+M)t_{y-1}}- e^{-(F+M)t_y})(L_{\infty}(1-e^{-Kt_y}))^3$$

Now $L_\infty$, $K$, $M$, and $N_0$ are all parameters we'd need to fit (we're going to assume that's already been done) and therefore the only free variable is $F$ - our management parameter. So $V(F)$ is just a function of our fishing mortality. As a result we can go ahead and plot it! With $L_\infty=1$, $K=0.1$, $M=0.1$, and $N_0=1$ we get Fig. \ref{fig:value_v_F}
\newline

\begin{figure}[h!] 
  \includegraphics[width=\linewidth]{notebooks/SimpleOptimization/value_v_F.png}
  \caption{Value versus Fishing Mortality}
  \label{fig:value_v_F}
\end{figure}

And if we walk through this example we can see why optimization is so important. Start at $F=0$ we get an obvious result - if you don't fish you don't derive any value from the fishery. Then as we begin to apply some fishing pressure $F>0$ we can see that the value starts rising rapidly - also expected. However something really interesting happens once we cross $F\approx 0.2$ the value starts to go down! What's going on here? Well it turns out that after a certain point ramping up the fishing pressure starts to kill so many \textit{young} fish that fish aren't able to grow up to the big weights that are more valuable to the fishery. As a result the fishery actually becomes poorer as we put more effort in! More work does not mean more rewards because that extra work actually changes the very demographics of the underlying stock. Super interesting! 
\newline

Alright, so as we said this was a pretty simple example. Part of what made it so simple was our "management" was parametrized by a single variable $F$. This meant that we could just plot out all of the options and choose the best one. In theory this brute force approach can always work. While as you get to parametrizations that have more 3 parameters you lose the ability to plot things out you can still run through all of the options and pick the best one. However there's a practical problem that quickly arises and its known as the curse of dimensionality. 
\newline

To illustrate suppose you start with a single parameter case like ours, but one where the parameter is categorical (i.e. it takes on a specific set of values). Maybe that parameter can take on 10 values. This in turn means you need only search $10$ different cases to find your optimal case for sure. 
\newline

Alright, now suppose that you add another 10 value parameter. How many options do you have to search? Well for each of the original 10 values of our first parameter we need to search the other 10 from this new parameter. So that's $10\bullet 10=100$ different cases. If we add another parameter it becomes $10\bullet 10\bullet 10=1000$ values. You can probably see where this is going... in general $N$ such parameters means $10^N$ cases!
\newline

This gets computationally inefficient really really fast. So when folks optimize really large problems (with loads of parameters) they instead have to turn to things besides brute force search.
\newline

Generally speaking there are two categories - analytic and meta-heuristic. While going into either case would take up far too much room here the basic idea behind each is to use knowledge about the solutions you've already got to be clever about the next case you choose so you can quickly find better values without having to search the \textit{whole} space. But this doesn't come for free as most optimization algorithms can't guarantee optimality. Instead you have to carefully understand their drawbacks and gotchas so that you can make it highly probably that a global maximum (or something very close to it) will be found. The point - know your optimizer! 
\newpage

\noindent \rule{\textwidth}{0.5pt} 
\begin{figure}[h!] 
  \includegraphics[width=\linewidth]{drawings/high_level.png}
  \caption{30,000ft View}
  \label{fig:high_level}
\end{figure}
\newline
\rule{\textwidth}{0.5pt} 
\vspace{5pt}

Hopefully at this point it's at least somewhat clear how this big problem breaks up into individual pieces. We've got models that drive our optimization, instrumentation creating ground truth data used to fit those models, and then the optimization itself. Put these together and you can find the right set of fishing parameters to maximize the long term value of the fishery. 
\newline

However it should also be clear how all of this is \textit{entirely} dependent on the models used and the fishing parameters chosen. So the question becomes - what can go wrong with the modeling itself? We turn to that next. 


\newpage


\section{Risky Business}
\newpage

\section{Unparalleled Progress}
\newpage

\section{Cartography}


Cite:
textbook
\bibliographystyle{plain}
\bibliography{reference}
\end{document}